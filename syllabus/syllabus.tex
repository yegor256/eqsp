% (The MIT License)
%
% Copyright (c) 2022 Yegor Bugayenko
%
% Permission is hereby granted, free of charge, to any person obtaining a copy
% of this software and associated documentation files (the 'Software'), to deal
% in the Software without restriction, including without limitation the rights
% to use, copy, modify, merge, publish, distribute, sublicense, and/or sell
% copies of the Software, and to permit persons to whom the Software is
% furnished to do so, subject to the following conditions:
%
% The above copyright notice and this permission notice shall be included in all
% copies or substantial portions of the Software.
%
% THE SOFTWARE IS PROVIDED 'AS IS', WITHOUT WARRANTY OF ANY KIND, EXPRESS OR
% IMPLIED, INCLUDING BUT NOT LIMITED TO THE WARRANTIES OF MERCHANTABILITY,
% FITNESS FOR A PARTICULAR PURPOSE AND NONINFRINGEMENT. IN NO EVENT SHALL THE
% AUTHORS OR COPYRIGHT HOLDERS BE LIABLE FOR ANY CLAIM, DAMAGES OR OTHER
% LIABILITY, WHETHER IN AN ACTION OF CONTRACT, TORT OR OTHERWISE, ARISING FROM,
% OUT OF OR IN CONNECTION WITH THE SOFTWARE OR THE USE OR OTHER DEALINGS IN THE
% SOFTWARE.

\documentclass[nobrand,anonymous,nodate,nosecurity]{huawei}
\usepackage{href-ul}
\begin{document}

{\sffamily{\bfseries\Large Ensuring Quality in Software Projects}\\
Series of lectures by \href{https://www.yegor256.com}{Yegor Bugayenko}\\
% to students of \href{https://innopolis.university/en/}{Innopolis University} in 2021,\\
% and \href{https://www.youtube.com/playlist?list=PLaIsQH4uc08woJKRAA7mmjs9fU0jeKjjM}{video recorded}}

% The entire set of slide decks is in \href{https://github.com/yegor256/ssd16}{yegor256/ssd16} GitHub repository.

\begin{abstract}
The course is a series of loosely coupled pieces of advice related to quality of software development.
Pragmatic programmers may listen to them if they don't want
to tolerate chaos in their projects. The course is not only about
coding practices, but also about static analysis, test coverage,
bug tracking, dependency and artifact management, build automation,
DevOps, and many other things. If we don't do them right, they may
severely jeopardize the quality of the entire project.
\end{abstract}

% \section*{Introduction}

\textbf{Who is the teacher?}
I'm developing software for more than 30 years, being a hands-on programmer
(see my GitHub account: \href{https://github.com/yegor256}{@yegor256})
and a manager of other programmers. At the moment I'm a director
of an R\&D laboratory in Huawei. Our primary research focus is
software quality problems. You may find some lectures I've presented
at some software conferences on
\href{https://www.youtube.com/channel/UCr9qCdqXLm2SU0BIs6d_68Q}{my YouTube channel}.

\textbf{Why this course?}
In \href{https://www.youtube.com/watch?v=kPmbRkSWYnY}{one of my videos}
a few years ago I explained what I believe is killing
most software projects: it's the chaos they can't control. Most of us
programmers start projects full of enthusiasm and best intentions.
We are confident that this time the design will be solid, the code will
be clean, and our customers will be happy because there will be no bugs.
Eventually, sooner rather than later, the reality appears to be as bad
as it was in the previous project: the code is messy, the design resembles
spaghetti, and the bugs are unpredictable and hard to fix. We learn some
lessons, abandon the project, and start a new one, again with the best
intentions. But in a new project nothing changes.
Most programmers that I know run in this cycle for decades.
I believe, this course may help you not become one of them.

\textbf{What's the methodology?}
The course is a collection of individual cases not
closely connected to each other. Each lecture discusses a single open-source
GitHub repository. Each discussion highlights the mistakes made in the
repository and suggests improvements. Each lecture ends with a conclusion
and a formulated recomendation. The recomendations may help students
prevent and control chaos in their own future projects.

\newpage
\section*{Course Aims}

Prerequisites to the course (it is expected that a student knows this):

\begin{itemize}
\item How to use Git
\item How to code
\item How to design software
\item How to write automated tests
\item How to deploy
\end{itemize}

After the course a student \emph{hopefully} will know:

\begin{itemize}
\item How to prevent dependency hell
\item How to use static analysis
\item How to organize automated tests
\item How to design integration tests
\item How to maintain documentation
\item How to convince a manager that quality is important
\item How to organize bug tracking
\item How to do code reviews
\item How to configure pre-flight builds
\item How to deal with negligence of other programmers
\item How to practice DevOps being a programmer
\item How to hire a programmer
\item How to measure test coverage
\item How to argue with customers
\item How to discipline fellow programmers
\item How to protect yourself from chaos
\end{itemize}

% \newpage
% \section*{Learning Material}

% The following books are highly recommended to read (in no particular order):

% \begin{multicols}{2}\small\raggedright
% Len Bass et al., \emph{Software Architecture in Practice}\\[3pt]
% Paul Clements et al., \emph{Documenting Software Architectures: Views and Beyond}\\[3pt]
% \nospell{Karl Wiegers} et al., \emph{Software Requirements}\\[3pt]
% {\nospell{Alistair Cockburn}}, \emph{Writing Effective Use Cases}\\[3pt]
% {\nospell{Steve McConnell}}, \emph{Software Estimation: Demystifying the Black Art}\\[3pt]
% {Robert Martin}, \emph{Clean Architecture: A Craftsman's Guide to Software Structure and Design}\\[3pt]
% {Steve McConnell}, \emph{Code Complete}\\[3pt]
% {Frederick Brooks Jr.}, \emph{Mythical Man-Month, The: Essays on Software Engineering}\\[3pt]
% {David Thomas et al.}, \emph{The Pragmatic Programmer: Your Journey To Mastery}\\[3pt]
% {Robert C. Martin}, \emph{Clean Code: A Handbook of Agile Software Craftsmanship}\\[3pt]
% {\nospell{Grady Booch} et al.}, \emph{Object-Oriented Analysis and Design with Applications}\\[3pt]
% {\nospell{Bjarne Stroustrup}}, \emph{Programming: Principles and Practice Using C++}\\[3pt]
% {\nospell{Brett McLaughlin} et al.}, \emph{Head First Object-Oriented Analysis and Design: A Brain Friendly Guide to OOA\&D}\\[3pt]
% {David West}, \emph{Object Thinking}\\[3pt]
% {Eric Evans}, \emph{Domain-Driven Design: Tackling Complexity in the Heart of Software}\\[3pt]
% {Yegor Bugayenko}, \emph{Elegant Objects}\\[3pt]
% {Michael Feathers}, \emph{Working Effectively with Legacy Code}\\[3pt]
% {Martin Fowler}, \emph{Refactoring: Improving the Design of Existing Code}\\[3pt]
% {Erich Gamma et al.}, \emph{Design Patterns: Elements of Reusable Object-Oriented Software}\\[3pt]
% {Scott Meyers}, \emph{Effective C++: 55 Specific Ways to Improve Your Programs and Designs}\\[3pt]
% {\nospell{Elliotte Rusty Harold} et al.}, \emph{XML in a Nutshell, Third Edition}\\[3pt]
% {\nospell{Michael James Fitzgerald}}, \emph{Learning XSLT: A Hands-On Introduction to XSLT and XPath}\\[3pt]
% {Martin Fowler}, \emph{UML Distilled}\\[3pt]
% {\nospell{Anneke Kleppe} et al.}, \emph{MDA Explained: The Model Driven Architecture: Practice and Promise}\\[3pt]
% {C.J. Date}, \emph{An Introduction to Database Systems, 8th Edition}\\[3pt]
% {\nospell{Pramod Sadalage} et al.}, \emph{NoSQL Distilled: A Brief Guide to the Emerging World of Polyglot Persistence}\\[3pt]
% {\nospell{Jez Humble} et al.}, \emph{Continuous Delivery: Reliable Software Releases through Build, Test, and Deployment Automation}\\[3pt]
% {\nospell{Michael T. Nygard}}, \emph{Release It!: Design and Deploy Production-Ready Software}\\[3pt]
% {Leonard Richardson et al.}, \emph{RESTful Web APIs: Services for a Changing World}
% \end{multicols}

\end{document}
