% (The MIT License)
%
% Copyright (c) 2022 Yegor Bugayenko
%
% Permission is hereby granted, free of charge, to any person obtaining a copy
% of this software and associated documentation files (the 'Software'), to deal
% in the Software without restriction, including without limitation the rights
% to use, copy, modify, merge, publish, distribute, sublicense, and/or sell
% copies of the Software, and to permit persons to whom the Software is
% furnished to do so, subject to the following conditions:
%
% The above copyright notice and this permission notice shall be included in all
% copies or substantial portions of the Software.
%
% THE SOFTWARE IS PROVIDED 'AS IS', WITHOUT WARRANTY OF ANY KIND, EXPRESS OR
% IMPLIED, INCLUDING BUT NOT LIMITED TO THE WARRANTIES OF MERCHANTABILITY,
% FITNESS FOR A PARTICULAR PURPOSE AND NONINFRINGEMENT. IN NO EVENT SHALL THE
% AUTHORS OR COPYRIGHT HOLDERS BE LIABLE FOR ANY CLAIM, DAMAGES OR OTHER
% LIABILITY, WHETHER IN AN ACTION OF CONTRACT, TORT OR OTHERWISE, ARISING FROM,
% OUT OF OR IN CONNECTION WITH THE SOFTWARE OR THE USE OR OTHER DEALINGS IN THE
% SOFTWARE.

\documentclass[nobrand,anonymous,nodate,nosecurity]{huawei}
\usepackage{href-ul}
\usepackage{ffcode}
\begin{document}

{\sffamily{\bfseries\Large Ensuring Quality in Software Projects}\\
Series of lectures by \href{https://www.yegor256.com}{Yegor Bugayenko}
% to students of \href{https://innopolis.university/en/}{Innopolis University} in 2021,\\
% and \href{https://www.youtube.com/playlist?list=PLaIsQH4uc08woJKRAA7mmjs9fU0jeKjjM}{video recorded}}

% The entire set of slide decks is in \href{https://github.com/yegor256/ssd16}{yegor256/ssd16} GitHub repository.

\begin{abstract}
The course is a series of loosely coupled pieces of advice related to quality of software development.
Pragmatic programmers may listen to them if they don't want
to tolerate chaos in their projects. The course is not only about
coding practices, but also about static analysis, test coverage,
bug tracking, dependency and artifact management, build automation,
DevOps, and many other things. If we don't do them right, they may
severely jeopardize the quality of the entire project, no matter
how good are your algorithms.
\end{abstract}

% \section*{Introduction}

\textbf{What is the goal?}\\
The primary objective of the course is to explain by example how
quality in a software project can be put under control.

\textbf{Who is the teacher?}\\
I'm developing software for more than 30 years, being a hands-on programmer
(see my GitHub account: \href{https://github.com/yegor256}{@yegor256})
and a manager of other programmers. At the moment I'm a director
of an R\&D laboratory in Huawei. Our primary research focus is
software quality problems. You may find some lectures I've presented
at some software conferences on
\href{https://www.youtube.com/channel/UCr9qCdqXLm2SU0BIs6d_68Q}{my YouTube channel}.
I also published \href{https://www.yegor256.com/books.html}{a few books}
and wrote \href{https://www.yegor256.com/contents.html}{a blog} about software engineering
and object-oriented programming.

\textbf{Why this course?}\\
In \href{https://www.youtube.com/watch?v=kPmbRkSWYnY}{one of my videos}
a few years ago I explained what I believe is killing
most software projects: it's the chaos they can't control. Most of us
programmers start projects full of enthusiasm and best intentions.
We are confident that this time the design will be solid, the code will
be clean, and our customers will be happy because there will be no bugs.
Eventually, sooner rather than later, the reality appears to be as bad
as it was in the previous project: the code is messy, the design resembles
spaghetti, and the bugs are unpredictable and hard to fix. We learn some
lessons, abandon the project, and start a new one, again with the best
intentions. But in a new project nothing changes.
Most programmers that I know run in this cycle for decades.
I believe, this course may help you not become one of them.

\textbf{What's the methodology?}\\
The course is a collection of individual cases not
closely connected to each other. Each lecture discusses a single open-source
GitHub repository. Each discussion highlights technical decisions made in the
repository and explains them in details. Each lecture ends with a conclusion
and a formulated recomendation. The recomendations may help students
prevent and control chaos in their own future projects.

\newpage
\section*{Course Structure}

Prerequisites to the course (it is expected that a student knows this):

\begin{itemize}
\item How to use Git
\item How to code
\item How to design software
\item How to write automated tests
\item How to deploy
\end{itemize}

After the course a student \emph{hopefully} will understand:

\begin{itemize}
\item How to avoid code smells
\item How to convince a manager that quality is important
\item How to deal with negligence of other programmers
\item How to hire a programmer who cares
\item How to argue with customers about quality
\item How to discipline fellow programmers
\item How to protect yourself from chaos
\end{itemize}

Also, a student will be able to practice:

\begin{itemize}
\item Source control:
	\href{https://git-scm.com}{Git},
	\href{https://subversion.apache.org}{Subversion}
\item Build automation:
	\href{https://en.wikipedia.org/wiki/Make_(software)}{Make},
	\href{https://maven.apache.org}{Maven},
	\href{https://gradle.org}{Gradle},
	\href{https://gruntjs.com}{Grunt},
	\href{https://github.com/ruby/rake}{Rake}
\item Dependencies:
	\href{https://maven.apache.org}{Maven Central},
	\href{https://www.npmjs.com}{NpmJS},
	\href{https://rubygems.org}{RubyGems},
	\href{https://pypi.org/project/pip/}{PyPi}
\item Static analysis:
	\href{https://clang.llvm.org/extra/clang-tidy/}{Clang-Tidy},
	\href{https://spotbugs.github.io}{SpotBugs},
	\href{https://scan.coverity.com}{Coverity}
\item Style checking:
  \href{https://checkstyle.sourceforge.io}{Checkstyle},
  \href{https://pmd.github.io}{PMD},
  \href{https://rubocop.org}{Rubocop},
  \href{https://eslint.org}{Eslint},
  \href{https://www.qulice.com}{Qulice}
\item Automated tests:
	\href{https://junit.org/}{JUnit},
	\href{https://mochajs.org}{Mocha}
\item Integration tests:
  \href{https://cucumber.io}{Cucumber},
  \href{https://www.selenium.dev}{Selenium},
  \href{https://en.wikipedia.org/wiki/Cross-browser_testing}{Cross-Browser}
\item Performance testing:
  \href{https://jmeter.apache.org}{JMeter}
\item Mocking frameworks:
	\href{https://site.mockito.org}{Mockito},
	\href{https://github.com/powermock/powermock}{PowerMock}
\item Textual documentation:
	\href{https://en.wikipedia.org/wiki/Markdown}{Markdown},
	\href{https://en.wikipedia.org/wiki/Wiki}{Wiki},
	\href{https://en.wikipedia.org/wiki/LaTeX}{LaTeX},
  \href{https://en.wikipedia.org/wiki/Controlled_natural_language}{CNL}
\item Bug tracking:
	\href{github.com/}{GitHub},
	\href{https://www.atlassian.com/software/jira}{JIRA},
	\href{https://www.bugzilla.org}{Bugzilla}
\item Code reviews:
	\href{github.com/}{GitHub},
	\href{https://www.gerritcodereview.com}{Gerrit},
	\href{https://www.atlassian.com/software/crucible}{Crucible}
\item Test coverage:
	\href{https://www.eclemma.org/jacoco/}{JaCoCo},
	\href{codecov.io/}{Codecov}
\item Mutation coverage:
	\href{https://www.pitest.org}{PIT}
\item Property-based testing:
	\href{https://en.wikipedia.org/wiki/QuickCheck}{Quickcheck}
\item DevOps:
	\href{https://www.docker.com}{Docker},
	\href{https://www.heroku.com}{Heroku},
	\href{https://aws.amazon.com/}{AWS}
\item Pre-flight builds:
	\href{https://github.com/features/actions}{GitHub Actions},
	\href{https://www.jenkins.io}{Jenkins},
	\href{https://www.rultor.com}{Rultor}
\item Metrics:
	\href{https://www.sonarqube.org}{SonarQube},
	\href{https://codeclimate.com/}{CodeClimate},
	\href{https://www.jpeek.org}{jPeek}.
\end{itemize}

\newpage
\section*{Lectures}

\newcommand\github[1]{\href{https://github.com/#1}{#1}}

This is a list of cases that will be discussed at the lectures:

\begin{enumerate}
	\item Ruby style checking with Rubocop in \github{yegor256/sibit}
  \item XML style checking in \github{yegor256/xcop}
	\item Custom Checkstyle rule in \github{yegor256/qulice}
  \vspace{3pt}

  \item GitHub actions in \github{yegor256/fibonacci} and \github{jcabi/jcabi-xml}
	\item Code reviews in \github{objectionary/eo}
  \item Making and testing a new GitHub Action in \github{yegor256/latexmk-action}
  \vspace{3pt}

	\item Npm dependencies and Grunt in \github{objectionary/eoc}
	\item Managing Maven dependencies in \github{yegor256/takes} (with \href{https://github.com/renovatebot/renovate}{Renovate})
	\item Multi-module \ff{pom.xml} in \github{objectionary/eo}
  \vspace{3pt}

	\item Build automation with \ff{Makefile} in \github{yegor256/fibonacci}
	\item Build automation with Gradle in \github{objectionary/eo-intellij-plugin}
  \vspace{3pt}

	\item Making a Ruby gem in \github{zold-io/zold}
	\item Packaging for CTAN in \github{yegor256/iexec}
	\item Deploying to Maven Central in \github{yegor256/cactoos}
	\item Deploying Java app to Heroku in \github{yegor256/rultor}
	\item Deploying Java app to Dokku in \github{yegor256/jare}
	\item Deploying a Ruby web app in \github{yegor256/sixnines}
  \item Docker image releasing to the Hub in \github{yegor256/rultor-image}
  \item Reversive deployment to AWS EC2 in \github{yegor256/s3auth}
  \vspace{3pt}

  \item Documenting \ff{README.md} of a Java library in \github{yegor256/takes}

  \item Integration testing with Maven Invoker Plugin in \github{jcabi/jcabi-xml}
  \item Testing for thread-safety in \github{yegor256/cactoos}
	\item Fake GitHub and AWS S3 objects in \github{jcabi/jcabi-github} and \github{jcabi/jcabi-s3}
  \item Parametrized testing with YAML in \github{objectionary/eo}
	\item Integration testing against DynamoDB Local in \github{yegor256/rultor}
  \vspace{3pt}

	\item JaCoCo coverage control in \github{yegor256/cactoos}
  \item CNL for requirements specification in \github{yegor256/requs}
  \item Stress testing in \github{yegor256/takes}
  \item Credentials testing in \github{yegor256/0pdd}
  \item Headless in-browser testing with Selenium in \github{yegor256/jare}
  \item Testcontainers in \github{yegor256/threecopies}
  \item BDD with Cucumber in \github{cqfn/pdd}

  \item Hits-of-code and other metrics in \github{yegor256/cobench} and \github{yegor256/hoc}
  \item Calculating Java cohesion metrics in \github{cqfn/jpeek}
  \item Developers performance in \github{yegor256/cobench}
\end{enumerate}

Students are welcome to pick most interesting cases studies.

\newpage
\section*{Laboratory Classes}

A few following laboratory classes may support the course, where students
will be asked to solve some of these tasks (the most complex are at the bottom):

\begin{enumerate}
	\item Configure GitHub Action to publish code coverage to \href{https://www.codecov.io}{codecov}
	\item Make both JUnit4 and JUnit5 tests work inside one Java repository
	\item Create a new unit test in BDD style, using Cucumber or a similar framework
	\item Design a coding style guide (in Markdown) for your favorite language
	\item Configure GitHub Actions to deploy a new JavaScript library to \href{https://npmjs.org}{npmjs}
	\item Automate headless Selenium+Safari integration testing in a simple web app
	\item Automate mutation coverage publishing to \href{https://pages.github.com}{GitHub Pages}
	\item Configure build to fail if test coverage is lower than 80\%
	\item Create a new plugin for GitHub Actions to validate the layout of repository
	\item Merge a pull request improving coding style to a 10k+ GitHub library
	\item Create an automated test for Java+Hibernate+MySQL with \href{https://www.testcontainers.org}{testcontainers}
	\item Create GitHub Action to spell check \texttt{README.md} using \href{http://aspell.net}{aspell}
	\item Install Jenkins and configure it to merge branches on demand
	\item Configure GitHub Action to send a message to Telegram when the build fails
	\item Automate cross-browsing testing of a web app using \href{https://www.browserstack.com}{BrowserStack}
	\item Configure property-based testing in an existing repository
	\item In a existing Java project automated with Maven, replace Maven with Makefile
	\item Automate comparison of two static analyzers, find out which one is stronger
	\item Develop a Maven plugin to check the quality of POM file
	\item Create a new style checking rule for PMD, to prohibit the use of non-final classes
	\item Using \href{http://aspell.net}{aspell}, create a style checker validating grammar in Java comments
\end{enumerate}

There could be other tasks too.

\newpage
\section*{Grading}

Students may form groups of up to four people. Each group will present
their own public GitHub repository with a software module inside. The group
will make a presentation of the quality control mechanisms that are
present in the repository. They will have to explain during a 10-minutes
oral presentation with live GitHub demonstration via screen sharing:

\begin{itemize}
	\item How enabled quality ensuring mechanisms work?
	\item Why such mechanisms are in use?
	\item How they help ensure quality?
	\item How often they get activated?
	\item What are the drawbacks of them?
	\item What mechanism are not used and why?
\end{itemize}

Most probably, there will
be no more than 20\% of ``A'' marks, no more than 40\% of ``B,''
and the rest will go to ``C'' and ``D.'' However, this distribution is
not mandatory: if all students make excellent presentations, everybody
will get ``A.''

Attendance will be tracked at the lectures. If you attend more than 75\%
of all lectures, you will not get less than ``C''.

At the laboratory classes each group will have to complete three
home works and defend them verbally on-site.
A completion of less than two will give everybody in the group a negative point,
a completion of three --- will give a positive point; the point will be added
to the grade given by the lecturer.

Higher grades will be given for:

\begin{itemize}
	\item Better understanding of the reasons behind used mechanisms,
	\item How they help ensure quality,
	\item How often they get activated,
	and
	\item What are the drawbacks of them.
\end{itemize}

A retake exam is possible, following exactly the same procedure. However,
the highest mark most probably possible at the retake is ``C.''

Students are highly advised to discuss
their repositories and quality ensuring mechanisms with each other,
before the final exam, in order to understand their relative positions
and maybe trigger new ideas.

\newpage
\section*{Learning Material}

The following books are highly recommended to read (in no particular order):

\begin{multicols}{2}\small\raggedright
{\nospell{Steve McConnell}}, \emph{Software Estimation: Demystifying the Black Art}\\[3pt]
{Robert Martin}, \emph{Clean Architecture: A Craftsman's Guide to Software Structure and Design}\\[3pt]
{Steve McConnell}, \emph{Code Complete}\\[3pt]
{Frederick Brooks Jr.}, \emph{Mythical Man-Month, The: Essays on Software Engineering}\\[3pt]
{David Thomas et al.}, \emph{The Pragmatic Programmer: Your Journey To Mastery}\\[3pt]
{Robert C. Martin}, \emph{Clean Code: A Handbook of Agile Software Craftsmanship}\\[3pt]
{David West}, \emph{Object Thinking}\\[3pt]
{Yegor Bugayenko}, \emph{Code Ahead}\\[3pt]
{Michael Feathers}, \emph{Working Effectively with Legacy Code}\\[3pt]
{\nospell{Jez Humble} et al.}, \emph{Continuous Delivery: Reliable Software Releases through Build, Test, and Deployment Automation}\\[3pt]
{\nospell{Michael T. Nygard}}, \emph{Release It!: Design and Deploy Production-Ready Software}\\[3pt]
\end{multicols}

\end{document}
