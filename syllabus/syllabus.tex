% (The MIT License)
%
% Copyright (c) 2022 Yegor Bugayenko
%
% Permission is hereby granted, free of charge, to any person obtaining a copy
% of this software and associated documentation files (the 'Software'), to deal
% in the Software without restriction, including without limitation the rights
% to use, copy, modify, merge, publish, distribute, sublicense, and/or sell
% copies of the Software, and to permit persons to whom the Software is
% furnished to do so, subject to the following conditions:
%
% The above copyright notice and this permission notice shall be included in all
% copies or substantial portions of the Software.
%
% THE SOFTWARE IS PROVIDED 'AS IS', WITHOUT WARRANTY OF ANY KIND, EXPRESS OR
% IMPLIED, INCLUDING BUT NOT LIMITED TO THE WARRANTIES OF MERCHANTABILITY,
% FITNESS FOR A PARTICULAR PURPOSE AND NONINFRINGEMENT. IN NO EVENT SHALL THE
% AUTHORS OR COPYRIGHT HOLDERS BE LIABLE FOR ANY CLAIM, DAMAGES OR OTHER
% LIABILITY, WHETHER IN AN ACTION OF CONTRACT, TORT OR OTHERWISE, ARISING FROM,
% OUT OF OR IN CONNECTION WITH THE SOFTWARE OR THE USE OR OTHER DEALINGS IN THE
% SOFTWARE.

\documentclass[nobrand,anonymous,nodate,nosecurity]{huawei}
\usepackage{href-ul}
\begin{document}

{\sffamily{\bfseries\Large Ensuring Quality in Software Projects}\\
Series of lectures by \href{https://www.yegor256.com}{Yegor Bugayenko}
% to students of \href{https://innopolis.university/en/}{Innopolis University} in 2021,\\
% and \href{https://www.youtube.com/playlist?list=PLaIsQH4uc08woJKRAA7mmjs9fU0jeKjjM}{video recorded}}

% The entire set of slide decks is in \href{https://github.com/yegor256/ssd16}{yegor256/ssd16} GitHub repository.

\begin{abstract}
The course is a series of loosely coupled pieces of advice related to quality of software development.
Pragmatic programmers may listen to them if they don't want
to tolerate chaos in their projects. The course is not only about
coding practices, but also about static analysis, test coverage,
bug tracking, dependency and artifact management, build automation,
DevOps, and many other things. If we don't do them right, they may
severely jeopardize the quality of the entire project.
\end{abstract}

% \section*{Introduction}

\textbf{Who is the teacher?}
I'm developing software for more than 30 years, being a hands-on programmer
(see my GitHub account: \href{https://github.com/yegor256}{@yegor256})
and a manager of other programmers. At the moment I'm a director
of an R\&D laboratory in Huawei. Our primary research focus is
software quality problems. You may find some lectures I've presented
at some software conferences on
\href{https://www.youtube.com/channel/UCr9qCdqXLm2SU0BIs6d_68Q}{my YouTube channel}.
I also published \href{https://www.yegor256.com/books.html}{a few books}
and wrote \href{https://www.yegor256.com/contents.html}{a blog} about software engineering
and object-oriented programming.

\textbf{Why this course?}
In \href{https://www.youtube.com/watch?v=kPmbRkSWYnY}{one of my videos}
a few years ago I explained what I believe is killing
most software projects: it's the chaos they can't control. Most of us
programmers start projects full of enthusiasm and best intentions.
We are confident that this time the design will be solid, the code will
be clean, and our customers will be happy because there will be no bugs.
Eventually, sooner rather than later, the reality appears to be as bad
as it was in the previous project: the code is messy, the design resembles
spaghetti, and the bugs are unpredictable and hard to fix. We learn some
lessons, abandon the project, and start a new one, again with the best
intentions. But in a new project nothing changes.
Most programmers that I know run in this cycle for decades.
I believe, this course may help you not become one of them.

\textbf{What's the methodology?}
The course is a collection of individual cases not
closely connected to each other. Each lecture discusses a single open-source
GitHub repository. Each discussion highlights the mistakes made in the
repository and suggests improvements. Each lecture ends with a conclusion
and a formulated recomendation. The recomendations may help students
prevent and control chaos in their own future projects.

\newpage
\section*{Course Aims}

Prerequisites to the course (it is expected that a student knows this):

\begin{itemize}
\item How to use Git
\item How to code
\item How to design software
\item How to write automated tests
\item How to deploy
\end{itemize}

After the course a student \emph{hopefully} will understand:

\begin{itemize}
\item How to avoid code smells
\item How to convince a manager that quality is important
\item How to deal with negligence of other programmers
\item How to hire a programmer who cares
\item How to argue with customers about quality
\item How to discipline fellow programmers
\item How to protect yourself from chaos
\end{itemize}

Also, a student will be able to practice:

\begin{itemize}
\item Source control:
	\href{https://git-scm.com}{Git},
	\href{https://subversion.apache.org}{Subversion}
\item Build automation:
	\href{https://en.wikipedia.org/wiki/Make_(software)}{Make},
	\href{https://maven.apache.org}{Maven},
	\href{https://gradle.org}{Gradle},
	\href{https://gruntjs.com}{Grunt},
	\href{https://github.com/ruby/rake}{Rake}
\item Dependencies:
	\href{https://maven.apache.org}{Maven Central},
	\href{https://www.npmjs.com}{NpmJS},
	\href{https://rubygems.org}{RubyGems},
	\href{https://pypi.org/project/pip/}{PyPi}
\item Static analysis:
	\href{https://clang.llvm.org/extra/clang-tidy/}{Clang-Tidy},
	\href{https://spotbugs.github.io}{SpotBugs},
	\href{https://scan.coverity.com}{Coverity}
\item Style checking:
  \href{https://checkstyle.sourceforge.io}{Checkstyle},
  \href{https://pmd.github.io}{PMD},
  \href{https://rubocop.org}{Rubocop},
  \href{https://eslint.org}{Eslint},
  \href{https://www.qulice.com}{Qulice}
\item Automated tests:
	\href{https://junit.org/}{JUnit},
	\href{https://mochajs.org}{Mocha}
\item Integration tests:
  \href{https://cucumber.io}{Cucumber},
  \href{https://www.selenium.dev}{Selenium},
  \href{https://en.wikipedia.org/wiki/Cross-browser_testing}{Cross-Browser}
\item Textual documentation:
	\href{https://en.wikipedia.org/wiki/Markdown}{Markdown},
	\href{https://en.wikipedia.org/wiki/Wiki}{Wiki},
	\href{https://en.wikipedia.org/wiki/LaTeX}{LaTeX},
  \href{https://en.wikipedia.org/wiki/Controlled_natural_language}{CNL}
\item Bug tracking:
	\href{github.com/}{GitHub},
	\href{https://www.atlassian.com/software/jira}{JIRA},
	\href{https://www.bugzilla.org}{Bugzilla}
\item Code reviews:
	\href{github.com/}{GitHub},
	\href{https://www.gerritcodereview.com}{Gerrit},
	\href{https://www.atlassian.com/software/crucible}{Crucible}
\item Test coverage:
	\href{https://www.eclemma.org/jacoco/}{Jacoco},
	\href{https://cobertura.github.io/cobertura/}{Cobertura}
\item Mutation coverage:
	\href{https://www.pitest.org}{PIT}
\item DevOps:
	\href{https://www.docker.com}{Docker},
	\href{https://www.heroku.com}{Heroku},
	\href{https://aws.amazon.com/}{AWS}
\item Pre-flight builds:
	\href{https://github.com/features/actions}{GitHub Actions},
	\href{https://www.jenkins.io}{Jenkins},
	\href{https://www.rultor.com}{Rultor}
\end{itemize}

\newpage
\section*{Grading}

Students may form groups of up to four people. Each group will present
their own public GitHub repository with a software module inside. The group
will make a presentation of the quality control mechanisms that are
present in the repository. They will have to explain during a 10-minutes
oral presentation with live GitHub demonstration via screen sharing:

\begin{itemize}
	\item How enabled quality ensuring mechanisms work?
	\item Why such mechanisms are in use?
	\item How they help ensure quality?
	\item How often they get activated?
	\item What are the drawbacks of them?
	\item What mechanism are not used and why?
\end{itemize}

Groups will compete with each other for the grades. Totally, there will
be no more than 20\% of ``A'' marks, no more than 40\% of ``B,''
and the rest will go do ``C'' and ``D.'' Students are highly advised to discuss
their repositories and quality ensuring mechanisms with each other,
before the final exam, in order to understand their relative positions
and maybe trigger new ideas.

Higher grades will be given for:

\begin{itemize}
	\item Better understanding of the reasons behind used mechanisms,
	\item How they help ensure quality,
	\item How often they get activated,
	and
	\item What are the drawbacks of them.
\end{itemize}

A retake exam is possible, following exactly the same procedure. However,
the higher mark possible at the retake is ``C.''

\newpage
\section*{Learning Material}

The following books are highly recommended to read (in no particular order):

\begin{multicols}{2}\small\raggedright
{\nospell{Steve McConnell}}, \emph{Software Estimation: Demystifying the Black Art}\\[3pt]
{Robert Martin}, \emph{Clean Architecture: A Craftsman's Guide to Software Structure and Design}\\[3pt]
{Steve McConnell}, \emph{Code Complete}\\[3pt]
{Frederick Brooks Jr.}, \emph{Mythical Man-Month, The: Essays on Software Engineering}\\[3pt]
{David Thomas et al.}, \emph{The Pragmatic Programmer: Your Journey To Mastery}\\[3pt]
{Robert C. Martin}, \emph{Clean Code: A Handbook of Agile Software Craftsmanship}\\[3pt]
{David West}, \emph{Object Thinking}\\[3pt]
{Yegor Bugayenko}, \emph{Code Ahead}\\[3pt]
{Michael Feathers}, \emph{Working Effectively with Legacy Code}\\[3pt]
{\nospell{Jez Humble} et al.}, \emph{Continuous Delivery: Reliable Software Releases through Build, Test, and Deployment Automation}\\[3pt]
{\nospell{Michael T. Nygard}}, \emph{Release It!: Design and Deploy Production-Ready Software}\\[3pt]
\end{multicols}

\end{document}
